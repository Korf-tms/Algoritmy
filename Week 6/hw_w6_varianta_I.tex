\documentclass[12pt,oneside]{article}
\usepackage{algpseudocode}
\usepackage{listings}
\usepackage[ruled]{algorithm}
\usepackage[czech]{babel}

\usepackage[a4paper,margin=2cm,head=0.5cm,foot=0.5cm]{geometry}
\usepackage{amsmath}
\usepackage{amsfonts}
\usepackage{amssymb}

\usepackage{color}

\definecolor{mygreen}{rgb}{0,0.6,0}
\definecolor{mygray}{rgb}{0.5,0.5,0.5}
\definecolor{mymauve}{rgb}{0.58,0,0.82}

% Taken from https://gist.github.com/mpdehnel/cc43e89a09c18ef6a602
\lstset{ %
	backgroundcolor=\color{white},   % choose the background color; you must add \usepackage{color} or \usepackage{xcolor}
	basicstyle=\footnotesize\ttfamily,        % the size of the fonts that are used for the code
	breakatwhitespace=false,         % sets if automatic breaks should only happen at whitespace
	breaklines=true,                 % sets automatic line breaking
	captionpos=b,                    % sets the caption-position to bottom
	commentstyle=\color{mygreen},    % comment style
	deletekeywords={...},            % if you want to delete keywords from the given language
	escapeinside={\%*}{*)},          % if you want to add LaTeX within your code
	extendedchars=true,              % lets you use non-ASCII characters; for 8-bits encodings only, does not work with UTF-8
	frame=single,                    % adds a frame around the code
	keepspaces=true,                 % keeps spaces in text, useful for keeping indentation of code (possibly needs columns=flexible)
	keywordstyle=\color{blue},       % keyword style
	language=Python,                 % the language of the code
	otherkeywords={*,...},            % if you want to add more keywords to the set
	numbers=left,                    % where to put the line-numbers; possible values are (none, left, right)
	numbersep=5pt,                   % how far the line-numbers are from the code
	numberstyle=\tiny\color{mygray}, % the style that is used for the line-numbers
	rulecolor=\color{black},         % if not set, the frame-color may be changed on line-breaks within not-black text (e.g. comments (green here))
	showspaces=false,                % show spaces everywhere adding particular underscores; it overrides 'showstringspaces'
	showstringspaces=false,          % underline spaces within strings only
	showtabs=false,                  % show tabs within strings adding particular underscores
	stepnumber=2,                    % the step between two line-numbers. If it's 1, each line will be numbered
	stringstyle=\color{mymauve},     % string literal style
	tabsize=2,                       % sets default tabsize to 2 spaces
	title=\lstname                   % show the filename of files included with \lstinputlisting; also try caption instead of title
}

%„“ 

\title{Domácí úkol k cvičení číslo 6}
\date{\today}

\begin{document}
	\maketitle
	
	\section{Graf ze souboru}
	
	Finálním cílem je vyrobit si šikovnou reprezentaci grafu.
	Budeme používat obvyklou definici grafu jako dvojice množiny vrcholů a hran a budeme chtít mít možnost sestavit jeho matici sousednosti a později i incidencí matici vrcholů a hran.
	Výsledné struktury později použijeme pro procházení do hloubky a šířky.
	
	\subsection{Vstup}
	
	Graf dostaneme zadaný textovým souborem, kde každý řádek bude ve formátu 
	\begin{verbatim}
		vrchol: seznam sousedních vrcholů oddělený čárkou a mezerou
	\end{verbatim}
	Vrcholy jsou označené čísly, ne nutně po sobě jdoucími.  
	Například:
		\begin{verbatim}
		1: 5, 6
		5: 1, 6
		6: 1, 5
	\end{verbatim}
	je graf o třech vrcholech $\{1, 5, 6\}$ a třech hranách $\{(1, 5), (5, 6), (1, 6)\}$ a lze nakreslit jako trojúhelník.
	
	Budou nás zajímat pouze neorientované grafy, ve kterých je hrana $(v_i, v_j)$ totéž jako $(v_j, v_i)$.
	Grafy by pak bylo možné zapisovat i úsporněji.
	Tj. následující soubor popisuje stejný graf
		\begin{verbatim}
		1: 5, 6
		5: 6
	\end{verbatim}
	a měl by mít i stejnou reprezentaci pomocí seznamu sousedů.
	Všimněte si, že do grafu je pak třeba přidávat i vrcholy, které jsou napravo od dvojtečky, v tomto případě vrchol \verb|6|.
	
	Povolíme si i hrany, které začínají a končí ve stejném vrcholu, tj. \verb*|1:1| je ok.
	Vrcholy, které nejsou v žádné hraně, akceptovat nebudeme a budeme předpkládat, že je nedostaneme na vstup, tj. \verb*|1:| není ok vstup.
	
	\subsection{Implementace grafu}
	
	Pro implementaci grafu použijte \verb|std::unordered_map|. Je to lepší přístup než skrze pole/vektor.
	Pro graf se hodí \verb|unordered_map<int, vector<int>>|, kde v prvním argumentu budou vrcholy a ve druhém bude vektor jeho sousedů, stejně jako na cvičení.
	
	Při procházení souboru a zapisování grafu se bude hodit funkce \verb*|std::find| pro určení, jestli hrana už v grafu je anebo jestli je třeba ji přidat.
	
	Pro přidání hrany, pokud už v grafu není, stačí použít operátor \verb*|[]|:
	\begin{verbatim}
		graph.adjacency_list[vertex_1].emplace_back(vertex_2);
	\end{verbatim}
	
	\paragraph{Matice sousednosti}
	Až budete mít hotové parsování souboru a uložení grafu, tak implementujte funkci která vytvoří \textit{matici sousednosti} grafu.
	Tohle by mělo být zadarmo ze cvičení.
	\textit{Matice sousednosti} $M$ je čtvercová matice o velikosti $n_v \times n_v$, kde $n_v$ je počet vrcholů, taková, že
	\begin{align}
		M_{ij} =  \begin{cases}
			1, & \text{pokud existuje hrana mezi vrcholy $v_i, v_j$}, \\
			0, & \text{jinak}.
		\end{cases}
	\end{align}
	Pro matici sousednosti tedy bude nutné vybrat si nejaké očíslování množiny vrcholů, implementovatelné třeba zase jako \verb|std::unordered_map|.
	
	
	\paragraph{Testy}
	Svůj program vyzkoušejte na několika ruzných rozumně malých grafech.
	Například:
	\begin{verbatim}
	7: 10, 111, 3
	10: 7, 111, 3
	111: 7, 10, 3
	3: 7, 10, 111
	\end{verbatim}
	Odpovídá kompletnímu grafu(=mezi každými dvěmi vrcholy je hrana) na čtyrech vrcholech a jeho matice sousednosti by tedy měla mít jedničku na každé pozici kromě diagonály.
	Jeho úsporná varianta
	\begin{verbatim}
	7: 10, 111, 3
	10: 111, 3
	111: 3
	\end{verbatim}
	by měla vést ke stejným výsledkům.
	
	Graf zadaný jako
	\begin{verbatim}
	1: 22, 55555
	22: 1, 333
	333: 22, 4444
	4444: 333, 55555
	55555: 1, 4444
	\end{verbatim}
	Odpovídá kružnici/pětiúhelníku a jeho matice sousednosti bude mít jedničky jen nad a pod diagonálou a v levém dolním a pravém horním rohu.
	Úspornější zápis téhož grafu je:
	\begin{verbatim}
	1: 22
	22: 333
	333: 4444
	4444: 55555
	55555: 1
\end{verbatim}


	Na náledující stránce je návrh struktury a některé užitečné funkce.
	
	\newpage
	\begin{lstlisting}
  struct Graph {
		unordered_map<int, vector<int>> adjacency_list;
		unordered_map<int, int> vertex_to_mat_index;
		int no_of_vertices;
		int no_of_edges;
    };
		
		
		void printGraph(const Graph& graph) {
			std::cout << "Graph:\n";
			for (const auto& [vertex, neighbours] : graph.adjacency_list) {
				std::cout << vertex << ": ";
				for( int neighbour : neighbours) {
					std::cout << neighbour << " ";
				}
				std::cout << "\n";
			}
			std::cout << std::endl;
		}
		
		
		Graph readGraphFromFile(const string& filename) {
			Graph graph;
			// TODO parsing goes here
			return graph;
		}
		
		
		vector<vector<int>> createAdjacencyMatrix(const Graph& graph) {
			int n = graph.no_of_vertices;
			vector<vector<int>> matrix(n, vector<int>(n, 0));
			// matrix construction goes here
			return matrix;
		}
		
		void constructIndexingMap(Graph& graph) {
			int index = 0;
			unordered_map<int, int> res;
			for (const auto& kv: graph.adjacency_list) {
				res[kv.first] = index;
				index += 1;
			}
			graph.vertex_to_mat_index = res;
		}
		
		
		void printGraphIndexingMap(const Graph& graph) {
			std::cout << "Graph vertex to index map:\n";
			for (const auto& [vertex, index] : graph.vertex_to_mat_index) {
				std::cout << vertex << ": " << index << ", ";
			}
			std::cout << std::endl;
		}
		
		void printMatrix(const vector<vector<int>>& matrix) {
			for (auto& line: matrix) {
				for(int val: line) {
					std:: cout << val << " ";
				}
				std::cout << "\n";
			}
		}
	\end{lstlisting}
	
\end{document}
