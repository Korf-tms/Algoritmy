\documentclass{article}
\usepackage{amsmath}
\usepackage[colorlinks=true]{hyperref}
\usepackage{listings}
\usepackage[czech]{babel}
% \usepackage[a4paper,margin=2cm,head=0.5cm,foot=0.5cm]{geometry}
\usepackage{algorithm}
\usepackage{algpseudocode}

\title{Domácí úkol k cvičení číslo 8}
\author{}
\date{\today}

\begin{document}
	
\maketitle

\section*{Generování permutací - Levitin str. 145}
Permutace budeme generovat pomocí čísel se šipkami, např. $\overset{\rightarrow}{5}, \overset{\leftarrow}{7} $.
Číslo se šipkou \( k \) je \textit{mobilní}, pokud jeho šipka ukazuje na menší číslo vedle něj.
Například, pro permutaci
\[ \overset{\leftarrow}{3} \overset{\leftarrow}{2} \overset{\rightarrow}{4} \overset{\rightarrow}{1}, \]
je prvek 4 mobilní, zatímco 3, 2 a 1 nejsou.
S použitím pojmu mobilního prvku můžeme dát následující popis algoritmu Johnson-Trotter pro generování permutací.

\begin{algorithm}
	\caption{JohnsonTrotter($n$)}
	\begin{algorithmic}[1]
		\Procedure{JohnsonTrotter}{$n$}
		\State Inicializuj první permutaci jako \( \overset{\leftarrow}{1} \overset{\leftarrow}{2} \ldots \overset{\leftarrow}{n} \)
		\While{předchozí permutace obsahuje mobilní prvek}
		\State Najdi její největší mobilní prvek $k$
		\State Proveďte výměnu prvku $k$ s prvkem, na který ukazuje jeho šipka
		\State Otočte směry šipek všech prvků, které jsou větší než $k$
		\State Přidejte novou permutaci do seznamu
		\EndWhile
		\EndProcedure
	\end{algorithmic}
\end{algorithm}

\textbf{Při implementaci proveďte rozumné rozdělení programu do funkcí.}
Použijte adekvátní datové struktury, ať vyrobíte čitelný program.

Zde je aplikace tohoto algoritmu pro \( n = 3 \):
\[
\overset{\leftarrow}{1} \; \overset{\leftarrow}{2} \; \overset{\leftarrow}{3} \quad
\overset{\leftarrow}{1} \; \overset{\leftarrow}{3} \; \overset{\leftarrow}{2} \quad
\overset{\leftarrow}{3} \; \overset{\leftarrow}{1} \; \overset{\leftarrow}{2} \quad
\overset{\rightarrow}{3} \; \overset{\leftarrow}{2} \; \overset{\leftarrow}{1} \quad
\overset{\leftarrow}{2} \; \overset{\rightarrow}{3} \; \overset{\leftarrow}{1} \quad
\overset{\leftarrow}{2} \; \overset{\leftarrow}{1} \; \overset{\rightarrow}{3}.
\]
Srovejte výstup Vašeho programu s tím, co dostanete pomocí \verb|std::next_permution|.
Srovnejte rychlosti pro velká $n$.


\end{document}