\documentclass{article}
\usepackage{amsmath}
\usepackage[colorlinks=true]{hyperref}
\usepackage{listings}
\usepackage[czech]{babel}
% \usepackage[a4paper,margin=2cm,head=0.5cm,foot=0.5cm]{geometry}

\title{Domácí úkol k cvičení číslo 7}
\author{}
\date{\today}

\begin{document}
	
	\maketitle
	
	\section*{Popis 8-Puzzle}
	
	8-Puzzle je skládačka, která se hraje na mřížce o rozměrech 3x3 s osmi čtvercovými dlaždicemi označenými čísly 1 až 8 a jedním prázdným místem.
	Cílem je přeuspořádat dlaždice tak, aby byly ve správném pořadí podle řádků.
	\begin{table}[h]
	\begin{center}
		\begin{tabular}{|c|c|c|}
			\hline
			& 1 & 2 \\ \hline
			3 & 4 & 5 \\ \hline
			6 & 7 & 8 \\ \hline
		\end{tabular}
	\end{center}
\caption{Cílová konfigurace}
\end{table}
	Je povoleno posouvat dlaždice buď horizontálně, nebo vertikálně do prázdného místa.
	Následující diagram ukazuje posloupnost možných tahů:
	 
 \begin{center}
 	\begin{tabular}{cccc}
 		\begin{tabular}{|c|c|c|}
 			\hline
 			1 & 3 & \\ \hline
 			4 &  8& 5 \\ \hline
 			7 & 2 & 6 \\ \hline
 		\end{tabular}
 		&
 		$\rightarrow$
 		\begin{tabular}{|c|c|c|}
 			\hline
 			1 & & 3 \\ \hline
 			4 & 8 & 5 \\ \hline
 			7 & 2 & 6 \\ \hline
 		\end{tabular}
 		&
 		$\rightarrow$
 		\begin{tabular}{|c|c|c|}
 			\hline
 			1 & 8 & 3 \\ \hline
 			4 &  &5 \\ \hline
 			7 & 2 & 6 \\ \hline
 		\end{tabular}
 		&
 		$\rightarrow$
 		\begin{tabular}{|c|c|c|}
 			\hline
 			1 & 8 & 3 \\ \hline
 			4 & 2 &5 \\ \hline
 			7 &  &6 \\ \hline
 		\end{tabular}
 	\end{tabular}
 \end{center}
	
	\subsection*{Zadání úkolu}
	
	Vaším úkolem je implementovat algoritmus pro řešení 8-Puzzle pomocí prohledávání do hloubky (DFS).
	Vaše implementace by měla najít vzít vstupní konfiguraci a pomocí DFS ukázat, jestli z ní lze dojít do cílové configurace.
	
	Vše implementujte pro obecné $n\times n$ schéma, v \verb| main|u pak program vyzkoušejte pro několik $3\times 3$ vstupních konfigurací.
\end{document}