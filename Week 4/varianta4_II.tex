\documentclass[12pt,oneside]{article}
\usepackage{listings}
\usepackage[ruled]{algorithm2e}
\usepackage[czech]{babel}


\usepackage[a4paper,margin=2cm,head=0.5cm,foot=0.5cm]{geometry}
\usepackage{amsmath}
\usepackage{amsfonts}
\usepackage{amssymb}

\usepackage{color}

\definecolor{mygreen}{rgb}{0,0.6,0}
\definecolor{mygray}{rgb}{0.5,0.5,0.5}
\definecolor{mymauve}{rgb}{0.58,0,0.82}

% Taken from https://gist.github.com/mpdehnel/cc43e89a09c18ef6a602
\lstset{ %
	backgroundcolor=\color{white},   % choose the background color; you must add \usepackage{color} or \usepackage{xcolor}
	basicstyle=\footnotesize\ttfamily,        % the size of the fonts that are used for the code
	breakatwhitespace=false,         % sets if automatic breaks should only happen at whitespace
	breaklines=true,                 % sets automatic line breaking
	captionpos=b,                    % sets the caption-position to bottom
	commentstyle=\color{mygreen},    % comment style
	deletekeywords={...},            % if you want to delete keywords from the given language
	escapeinside={\%*}{*)},          % if you want to add LaTeX within your code
	extendedchars=true,              % lets you use non-ASCII characters; for 8-bits encodings only, does not work with UTF-8
	frame=single,                    % adds a frame around the code
	keepspaces=true,                 % keeps spaces in text, useful for keeping indentation of code (possibly needs columns=flexible)
	keywordstyle=\color{blue},       % keyword style
	language=Python,                 % the language of the code
	otherkeywords={*,...},            % if you want to add more keywords to the set
	numbers=left,                    % where to put the line-numbers; possible values are (none, left, right)
	numbersep=5pt,                   % how far the line-numbers are from the code
	numberstyle=\tiny\color{mygray}, % the style that is used for the line-numbers
	rulecolor=\color{black},         % if not set, the frame-color may be changed on line-breaks within not-black text (e.g. comments (green here))
	showspaces=false,                % show spaces everywhere adding particular underscores; it overrides 'showstringspaces'
	showstringspaces=false,          % underline spaces within strings only
	showtabs=false,                  % show tabs within strings adding particular underscores
	stepnumber=2,                    % the step between two line-numbers. If it's 1, each line will be numbered
	stringstyle=\color{mymauve},     % string literal style
	tabsize=2,                       % sets default tabsize to 2 spaces
	title=\lstname                   % show the filename of files included with \lstinputlisting; also try caption instead of title
}


\title{Domácí úkol k cvičení číslo 4}

\begin{document}
	\maketitle
	\section{Příklady ke složitosti}
	\subsection{Rozhodnětě a \underline{zdůvodněte}, jestli platí následující tvrzení:}
		\begin{enumerate}
		\item Pro funkci $f(n) = 8n^2 + 7n + 6$ platí:
		\begin{align}
			f(n) \in \Omega(n^2), \\
			f(n) \in \Theta(n^2).
		\end{align}
		\item Pro funkci $g(n) = 4n\log(8n) + n + 10$ platí:
		\begin{align}
			g(n) \in O(n^2), \\
			g(n) \in \Theta(n\log n).
		\end{align}
		\item Pro funkci $h(n) = 3n^2 + 10^4n+  \pi $ platí:
		\begin{align}
			h(n) \in O(n^3) ,\\
			h(n) \in \Theta(n^3).
		\end{align}
	\end{enumerate}
	Za dostatečné zdůvodnění se považuje výpočet odpovídající limity nebo ukázka toho, že lze nalézt $c_1, c_2$ a $n_0$ v definicích $\Omega, \Theta, O$, viz přednáška a doporučená literatura.
	Dostatečné zdůvodnění by mělo obsahovat komentář v přirozeném jazyce, alespoň na úrovni: ,,Tvrzení platí/neplatí protože:\dots``
	
	\subsection{Určete a zdůvodněte jaká je složitost daného algoritmu}
	\begin{algorithm}[ht]
		\caption{What does this do?}
		\DontPrintSemicolon
		//Input: $n\times (n+1)$ matrix $A[0\dots n-1; 0 \dots n]$ of real numbers \;
		\For{$i=0, \dots, n-2$}{
			\For{$j=i+1, \dots, n-1$}{
				\For{$k = i, \dots, n$}{
					$A[j, k] = A[j, k] - A[i, k]*A[j, i]/ A[i, i] $\;
				}
			}
		}
		\Return {$A$}
		\end{algorithm}
		Předpokládejme, že cena všech operací násobení, sčítání, odečítání a dělení je stejná, a že je nezávislá na velikosti čísel.
	Dále se nebudeme trápit dělením nulou.
	Zápis \verb|for| cyklu v algoritmu je myšlený tak, že se začíná v dolní mezi a jde se do horní meze včetně.
	
	Jak by šlo algoritmus snadno zefektivnit?
	
	\subsection{Vyřešte rekurentní rovnici}
	Najděte funkci $T$ takovou, že $T(0) = 1$ pro všechna $n \in \mathbb{N}$ (všechna kladná celá čísla) platí
	\begin{align}
		T(n) = 2 + T(n-1).
	\end{align}
	
	\section{Úloha k naprogramování}
	Finálním cílem je vyrobit si šikovnou reprezentaci grafu.
	Budeme používat obvyklou definici grafu jako dvojice  množiny vrcholů a hran a budeme chtít mít možnost sestavit jeho incideční matici a později i matici sousednosti vrcholů a hran.
	\subsection{Vstup}
	Graf dostaneme zadaný textovým souborem, kde každý řádek bude ve formátu \newline
	\verb|vrchol: seznam sousedních vrcholů oddělený čárkou a mezerou|.
	\newline
	Vrcholy jsou označené čísly, ne nutně po sobě jdoucími.  
	Například:
	\begin{verbatim}
		1: 5, 6
		5: 6
	\end{verbatim}
	je graf o třech vrcholech $\{1, 5, 6\}$ a třech hranách $\{(1, 5), (5, 6), (1, 6)\}$ a lze nakreslit jako trojúhelník.
	
	Pokud bude v textovém souboru nějaká hrana vícekrát, do grafu ji přidáme jen jednou.
	Tj. následující soubor popisuje stejný graf
	\begin{verbatim}
		1: 5, 6
		5: 1, 6
		6: 1, 5
	\end{verbatim}
	Povolíme si i hrany, které začínají a končí ve stejném vrcholu, tj. \verb*|1:1| je ok.
	Stejně tak akceptujeme i vrcholy, které nejsou v žádné hraně, tj. \verb*|1:| je ok.
	
	
	\subsection{Implementace grafu}
	Pro implementaci grafu použijte \verb*|std::unordered_map|. Je to lepší přístup než skrze pole/vektor, protože je rychlejší a umožňuje přístup a zápis typu \verb*|value = mapa[key]|.
	Pro graf se hodí  \verb*|unordered_map<int, vector<int>>|, kde v prvním argumentu budou vrcholy a ve druhém bude seznam/vektor jeho sousedů.
	
	Až budete mít hotové parsování souboru a konstrukci grafu, implementujte funkci, která vytvoří matici sousednosti grafu.
	
	Minimalistická kostra programu je na následující stránce, nenechte se omezovat a rozšiřte podle potřeby jak struktury tak funkce.
	
\newpage
	
\begin{lstlisting}
struct Graph {
	unordered_map<int, vector<int>> adjacency_list;
	int no_of_vertices;
	int no_of_edges;
};


void printGraph(const Graph& graph) {
	std::cout << "Graph:\n";
	for (const auto& [vertex, neighbours] : graph.adjacency_list) {
		std::cout << vertex << ": ";
		for( int neighbour : neighbours) {
			std::cout << neighbour << " ";
		}
		std::cout << "\n";
	}
	std::cout << std::endl;
}


Graph readGraphFromFile(const string& filename) {
	Graph graph;
	// parsing goes here
	return graph;
}


vector<vector<int>> createAdjacencyMatrix(const Graph& graph) {
	int n = graph.no_of_vertices;
	vector<vector<int>> matrix(n, vector<int>(n, 0));
	// matrix construction goes here
	return matrix;
}

void printMatrix(const vector<vector<int>>& matrix) {
	for (auto& line: matrix) {
		for(int val: line) {
			std:: cout << val << " ";
		}
		std::cout << "\n";
	}
}
\end{lstlisting}
	
	
\end{document}